
\subsubsection{DataInputStream}
\label{sec:uml:input:datainputstream}

在 Java 官方的 IO 包中,官方提供了 \lstline|DataInputStream| 这个类。
在 API 文档中的描述如下
\begin{quote}
  A data input stream lets an application read primitive Java data types from an underlying input stream in a machine-independent way. 
  An application uses a data output stream to write data that can later be read by a data input stream.

  DataInputStream is not necessarily safe for multithreaded access. Thread safety is optional and is the responsibility of users of methods in this class.
\end{quote}

这一段描述交代了两件事, \lstinline|DataInputStream| 是一个用于读取“原始”的Java数据,并且与硬件底层无关。第二件事指出了其不是线程安全的,线程安全由用户保证。

\lstinline|DataInputStream| 继承自 \FilterInputStream|, 同时实现了 \lstinline|DataInput| 接口。
\lstinline|DataInputStream| 的使用基本上就是以 \lstinline|DataInputStream var = new DataInputStream(new InputStream(..));| 的方式使用,类似
\lstinline|BufferedInputStream|类似。其提供了17个 read 函数的实例,分贝可以读取字节,布尔类型,字符类型,浮点类型,和整数类型,同是提供了读取一行,和跳过字节的功能。
同时还从 \lstinline|FilterInputStream| 继承了一些其他的方法。

%% 类图

%% 代码(主要的read)
