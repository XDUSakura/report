
\subsubsection{FSDataInputStream}
\label{sec:uml:input:fsdatainputstream}

\lstinline{FSDataInputStream} 这个类提供了对 \lstinline|java.io| 包中的 \lstinline|DataInputStream| 支持。
\lstinline{FSDataInputStream} 继承自 \lstinline|DataInputStream|,同时实现了 \lstinline|Seekable|, \lstinline|PositionedReadable|, \lstinline|ByteBufferReadable|, \lstinline|CanSetDropBehind|, \lstinline|CanSetReadahead| 等接口,
以提供搜索,随机读取,以字节的缓冲试读取,以及丢弃缓存等功能。

在 Hadoop 官方的 API 文档中,写着这样的一句话:
\begin{quote}
    Utility that wraps a FSInputStream in a DataInputStream and buffers input through a BufferedInputStream.
\end{quote}
意味着 \lstinline|FSDataInputStream| 融合了 \lstinline|DataInputStream| 和 \lstinline|FSInputStream|,同时提供了
\lstinline|BufferedInputStream| 的输入方式。

\lstinline|FSDataInputStream| 提供了一些列的扩展的读取方法与随机跳转的方法。读取的方法中,主要提供了基于 缓冲区的,和缓冲区池的
读取方法。随机读取与跳转提供了寻找跳转与抛弃跳转的功能。
\lstinline|FSDataInputStream| 再加上继承来的函数,其提供了一个对于分布式文件系统带来的输入流的“使用”。

%% 类图

%% 代码