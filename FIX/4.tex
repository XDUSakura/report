%% 批注四的回答

\subsection{批注四的回答}
\label{sec:fix:bl4}

针对第四个批注的内容,在回答之前首先需要介绍一下很多云服务提供商的文件存储服务。
诸如腾讯云,青云,阿里云,亚马逊云,微软的Azure等云服务提供商,会提供多种服务,包括云主机的业务,和文件存储的服务。
通常情况下,储存数据的选择有很多,可以使用数据库,无论是传统的数据库还是 NoSQL 的数据库,同时还可以选择“硬盘”。
云服务上提供的“硬盘”,通常有两种,一种是提供给虚拟主机使用的硬盘,与物理主机使用的硬盘类似。此外还有另一种,
存储服务,由于前者并不适用于很多业务场景,诸如一份大文件为需要被很多节点使用,所以大多是提供云服务的服务商,同时提供了
对象存储服务。下面是摘录自腾讯云\textit{对象存储服务}页面的一段介绍内容:
\begin{quote}
    对象存储(Cloud Object Storage)是面向企业和个人开发者提供的高可用,高稳定,强安全的云端存储服务。
    您可以将任意数量和形式的非结构化数据放入COS,并在其中实现数据的管理和处理。COS支持标准的Restful API接口,
    您可以快速上手使用,按实际使用量计费,无最低使用限制。
\end{quote}
很多提此类服务供商处于一定技术原因,会限制单个存储对象的大小。而这里的存储对象和传统的文件有类似的地方但是概念也不大相同。
在最新的亚马逊的官方文档中,存储服务已经更名为 \verb|Amazon Elastic File System|,EFS 为缩写。 在 EFS 的 FAQ 并没有见到
提及任何大小限制,同时有如下一段描述:
\begin{quote}
    Amazon EFS 文件系统可以自动将数据容量从 GB 级扩展到 PB 级,无需预置存储。
\end{quote}

简单概括,这个文件系统就是将对象存储服务的 API 封装成类似与文件系统的东西,也就是原生的。5G 大小的限制则是来自
对象存储服务的限制,目前应该是已经取消。