%% 批注二

\subsection{批注二的回答}
\label{sec:fix:bl2}

HDFS 与 Linux 系统中的 VFS 相类似。首先在 Linux 系统中, VFS 是 \verb|Virtual File System| 的缩写。 VFS 解决掉的问题不是说针对某一个特定的文件系统的实现,或者
文件系统应该如何实现。VFS 做的是 抽象的描述出了一个文件系统提供,尤其是对上层提供的 API 接口或者是所具有的功能有哪些。简而言之,描述了一个对于更高层次的结构来说,文件系统长得什么样子。
HDFS 类似的提拱了一些“接口描述”,或者更加形象使用 Haskell 中的类型类进行“比喻”。

由此可见 HDFS 是依赖与底层的文件系统实现,然后向高层用户提供相关内容的访问。而底层文件系统则相对“依赖” HDFS 中描绘的接口,以规范自身。

HDFS 同时做的一件事情是:将不同的文件系统对接到一起。正如在 Linux 中,我们可以将 Ext4 文件系统下的文件复制到 exFAT 文件系统中, HDFS 也提供了类似的支持。
本身两种不同文件系统会具有比较大的差异,而 HDFS 做的是从文件系统A中拿出东西,在放到文件系统B中。