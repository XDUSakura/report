%%%%%%%%%%%%%%%%%%%%%%%%%%%%%%%%%%%%%%%%%%%%%%%%%%%
%%%%%%%%%%%%%%%%%%%%%%%%%%%%%%%%%%%%%%%%%%%%%%%%%%%
%%
%%       Report of the analyzing for Hadoop
%%
%%       Copyright (C) XDUSakura 2017
%% 
%%       <pari passu>
%%       李约瀚 (John Lee) (Qinka) <qinka@live.com> <me@qinka.pro>
%%       康赣鹏
%%       乔新闻
%%       罗赛男
%%       杨瑞
%%
%%
%%%%%%%%%%%%%%%%%%%%%%%%%%%%%%%%%%%%%%%%%%%%%%%%%%%
%%%%%%%%%%%%%%%%%%%%%%%%%%%%%%%%%%%%%%%%%%%%%%%%%%%

%% 使用 XeTeX 编译


% 使用 CTeX 的 Report 模版
\documentclass{ctexrep}

%% 导言区
%% 所有需要添加的宏包,先创建新分支然后测试,最后等待负责的人,合并到主开发分支
%% 首先从 dev/base 牵出一个新的分支,加上所需要增加的导言区,然后再从此处牵出另一个新分支
%% 并与所需要导言区的代码部分进行合并,并测试,测试成功之后,将第二次添加的分支删去
%% 第一次添加的分支等待被合并回 dev/base 分支
%% 然后其他