\section{Environment}
\label{sec:environment}

Hadoop 发展到今天,整个项目已经不能简单的用“庞大”这样的词语来形容了。运行 Hadoop 的环境通常需要配置很久,
虽然依赖 Java 的跨平台的特性, Hadoop 在不同平台上移植不是难事,然而部署却需要花费不少功夫。
通常的步骤是先编写一堆配置文件,然后初始化 HDFS, 最后验证并启动 Hadoop。
好在 Hadoop 官方提供了 Dockerfile, 能比较容易的将 Hadoop 打包成 Docker镜像,然后运行。

代码分析使用的是 Hadoop 2.8.0 的源代码,代码是从官方在 GitHub 上的仓库 fork 出来的。
同时还运用到了 StartUML 等工具绘制 UML 类图等。

%% 结束
\endinput